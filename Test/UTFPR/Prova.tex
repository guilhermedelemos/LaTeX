\documentclass[a4paper, 11pt, addpoints]{exam}
%\documentclass[a4paper, 11pt, addpoints, answers]{exam}  % Desconmente esta linha, para ver as respostas, e comente a de cima
\usepackage{UTFPR}
\usepackage{listings} % Mostrar código-fonte
\usepackage[brazil]{babel}
\usepackage{multicol}


\setlength{\columnsep}{25pt}
\lstdefinestyle{js}{
    basicstyle=\ttfamily,
    breaklines=true,
    breakatwhitespace=true,
    tabsize=1,
    resetmargins=true,
    xleftmargin=0pt,
    frame=none
}




%%%%%%%%%%%%%%%%%%%%%%%%%%%%%%%%%%%%%%%%%%%%%%%%%%%%

\pointpoints{Ponto}{Pontos}

\begin{document}

\nomeProfessor{Nome Professor}
\nomeCurso{Tecnologia em Sistemas para Internet}
\nomeDisciplina{Nome disciplina}
\semestre{3}   % Deixe em branco se for para mais de um semestre (recuperação)
\dataDaProva{01/04/2015}
\tipoAvaliacao{Suficiência}

\info
\vspace{-1.5 cm}


%%%%%%%%%%%%%%%%%%%%%%%%%%%%%%%%%%%%%%%%%%%%%%%%%










%%AAG: tipos de questões:
%               - \begin{checkboxes}
%               - \begin{choices}
%               - \begin{oneparchoices}
%               - \begin{oneparcheckboxes}
% vide exam.cls 3779



\begin{questions}
\begin{multicols}{2}

%comente o \begin multicols para coluna única



%% entre [ ] coloca-se o peso da questão. 
\question[2] Descrição da Questão - ?

\begin{choices}
\CorrectChoice Alternativa a
\choice Alternativa b
\choice Alternativa c
\choice Alternativa d
\choice Alternativa e
\end{choices}

\question[2] Descrição da Questão 2 - ?

\begin{choices}
\CorrectChoice Alternativa a
\choice Alternativa b
\choice Alternativa c
\choice Alternativa d
\choice Alternativa e
\end{choices}



%% Modelo de questão com código - fonte
%\question[35] Criar um c�digo PHP que mostra a quantidade de aulas m�dias restantes para cada disciplina do terceiro per�odo e salve no arquivo \lq{}tsi\_3p.txt\rq{}, utilizando as seguintes vari�veis e considerando:

%\begin{itemize}
%\item De segunda a quinta s�o 5 horas aulas e sexta s�o 4 horas;
%\item Desprezando as divis�es de aulas pela semana;

%\end{itemize}

%\begin{lstlisting}[style=js]
%<?php
%$hoje = '2015-04-01';
%$fim  = '2015-07-03'; 
%$feriados = array('2015-04-21', '2015-05-01', '2015-09-07', '2015-10-12', '2015-11-15', '2015-04-02', '2015-04-03', '2015-04-20', '2015-06-04', '2015-06-05');
%$dis_quant_aula = array("TSI33A" => 4, "TSI33B" => 4, "TSI33C" => 5, "TSI33D" => 4, "TSI33E" => 3, "TSI33F" => 4);

%\end{lstlisting}





\end{multicols}
\end{questions}
\end{document}